% !TEX program = pdflatex
% Introduction to Communication System Cheat Sheet
\documentclass[UTF8,a4paper,10pt]{article}
\usepackage[UTF8,scheme=plain,linespread=.1]{ctex}
\usepackage[margin=.1in]{geometry}
\usepackage{multicol}
\setlength{\columnseprule}{.2pt}
\usepackage{amsmath,amssymb,mathrsfs,bm}
\allowdisplaybreaks[4]
\providecommand{\abs}[1]{\left\lvert#1\right\rvert}
\providecommand{\re}{\,\text{Re}\,}
\providecommand{\im}{\,\text{Im}\,}
\providecommand{\sgn}{\,\text{sgn}\,}
\providecommand{\sinc}{\,\text{sinc}\,}
\providecommand{\det}{\,\text{det}\,}
\usepackage{ulem}
\begin{document}
\scriptsize
\begin{multicols}{2}
    \noindent\textbf{Analog Comm Sys Architecture}: Source$\rightarrow$\uline{Modulator} (Transmitter)$\rightarrow$Channel$\rightarrow$\uline{Detector} (Receiver)$\rightarrow$User; \textbf{Digital Comm Sys Arch}: Source$\rightarrow$\uline{A/D converter$\rightarrow$Source encoder (Compress)$\rightarrow$Channel encoder (Add redundancy for err correction)$\rightarrow$Modulator} (Transmitter)$\rightarrow$Ch$\rightarrow$\uline{Detector$\rightarrow$Ch decoder$\rightarrow$Src decoder$\rightarrow$D/A converter} (Receiver)$\rightarrow$Usr\\
    \textbf{Classification of Sgnl}: \textbf{Analog}: continuous time \& value, \textbf{Discrete-time}: discrete time, continuous value, \textbf{Digital}: discrete time \& value; \textbf{Deterministic}: modeled as completely specified func of time, \textbf{Random}: take random value at any given time and modeled probabilistically; \textbf{Periodic}: $x(t+T_0)=x(t),t\in\mathbb{R}$ where $T_0$ -- period, \textbf{Aperiodic}: non-periodic\\
    \textbf{Comm sys performance metrics}: \textbf{Reliability}: SNR for analog sys, Bit err rate for digital sys; \textbf{Efficiency}: Band eff$=\frac{\text{bit rate}}{\text{bandwidth}}$bit/s/Hz, Energy eff$=\frac{\text{bit energy}}{\text{noise power spectral density}}$\\
    \textbf{Energy of sgnl}: $E=\int_{t_1}^{t_2}\abs{x(t)}^2\,dt$ (J); \textbf{Energy sgnl}: $\int_{-\infty}^{\infty}\abs{x(t)}^2\,dt<\infty$;; \textbf{Average power of sgnl}: $P=\frac{1}{t_2-t_1}\int_{t_1}^{t_2}\abs{x(t)}^2\,dt$ (W); \textbf{Power sgnl}: $0<\lim_{T\rightarrow\infty}\int_{-T/2}^{T/2}\abs{x(t)}^2\,dt<\infty$\\
    \textbf{Unit impulse (/delta) func $\delta(t)$}: $\int_{-\infty}^{+\infty}x(t)\delta(t-t_0)\,dt=x(t_0)$ where $x(t)$ -- any func continuous at $t_0=0$ (\textbf{sifting property}); \textbf{Properties}: $\int_{t_1}^{t_2}\delta(t-t_0)\,dt=1$ for $t_1<t_0<t_2$, $\delta(t-t_0)=0$ for $t\neq t_0$, $\delta(at)=\frac{\delta(t)}{\abs{a}}$, $\int_{t_1}^{t_2}x(t)\delta^{(n)}(t-t_0)\,dt=(-1)^nx^{(n)}(t_0)$ for $t_1<t_0<t_2$\\
    \textbf{Unit step func}: $u(t)=\int_{-\infty}^t\delta(\lambda)\,d\lambda=\left\{\begin{array}{ll}
        1,&t>0\\
        0,&t<0
    \end{array}\right.$\\
    \textbf{Unit rectangular pulse func}: $\Pi(t)=u(t+\frac{1}{2})-u(t-\frac{1}{2})$\\
    \textbf{Fourier Series (complex exp form)}: $x(t)=\sum_{n=-\infty}^{\infty}X_n\exp(jn2\pi f_0t),t_0\leq t<t_0+T_0$, where $f_0=\frac{1}{T_0}$, $X_n=\frac{1}{T_0}\int_{t_0}^{t_0+T_0}x(t)\exp(-jn2\pi f_0t)\,dt$; \textbf{(trigonometric form)}: $x(t)=\sum_{n=0}^{\infty}[A_n\cos(n2\pi f_0t)+B_n\sin(n2\pi f_0t)]$, where $A_n=\frac{2}{T}\int_{t_0}^{t_0+T_0}x(t)\cos(n2\pi f_0t)\,dt$, $B_n=\frac{2}{T_0}\int_{t_0}^{t_0+T_0}x(t)\sin(n2\pi f_0t)\,dt$\\
    \textbf{Fourier transform (FT) of a continuous-time sgnl}: $X(f)=\mathscr{F}[x(t)]=\int_{-\infty}^{\infty}x(\lambda)e^{-j2\pi f\lambda}\,d\lambda$, if $x(t)$ is absolutely integrable, i.e., $\int_{-\infty}^{\infty}\abs{x(t)}\,dt<\infty$; \textbf{Inverse Fourier transform}: $x(t)=\mathscr{F}^{-1}[X(f)]=\int_{-\infty}^{\infty}X(f)\exp(j2\pi ft)\,df$\\
    \textbf{Spectrum}: $X(f)=\abs{X(f)}\exp[j\angle X(f)]$\\
    \textbf{Parseval's Theorem}: $\int_{-\infty}^{\infty}x_1(t)x_2^*(t)\,dt=\int_{-\infty}^{+\infty}X_1(f)X_2^*(f)\,df$\\
    For energy sgnl, \textbf{Energy spectral density}: $G(f)=\abs{X(f)}^2$ (J/Hz), energy: $E=\int_{-\infty}^{\infty}\abs{x(t)}^2\,dt=\int_{-\infty}^{\infty}\abs{X(f)}^2\,df$; \textbf{Time-average autocorrelation func}: $\phi(\tau)=x(\tau)*x(-\tau)=\int_{-\infty}^{\infty}x(\lambda)x(\lambda+\tau)\,d\tau=\lim_{T\rightarrow\infty}\int_{-T}^Tx(\lambda)x(\lambda+\tau)\,d\tau$; \textbf{Relation b/w ESP \& autocor}: $G(f)=\mathscr{F}[\phi(t)]$\\
    For power sgnl: \textbf{Power spectral density (PSD) $S(f)$}: ???; \textbf{Power}: $P=\int_{-\infty}^{\infty}S(f)\,df=\langle x^2(t)\rangle$; \textbf{Time-average autocorrelation func}: $R(\tau)=\langle x(t)x(t+\tau)\rangle=\lim_{T\rightarrow\infty}\frac{1}{2T}\int_{-T}^Tx(t)x(t+\tau)\,dt$; \textbf{Relation between PSD \& autocor}: $S(\tau)=\mathscr{F}[R(\tau)]$\\
    \textbf{Properties of FT}: for $x_{1/2}(t)\leftrightarrow X_{1/2}(f)$, \textbf{Superposition}: $\mathscr{F}[a_1x_1(t)+a_2x_2(t)]=a_1X_1(f)+a_2X_2(f)$; \textbf{Scaling}: $\mathscr[X(at)]=\frac{1}{\abs{a}}X(\frac{f}{a})$; \textbf{Time shifting}: $\mathscr{F}[x(t-t_0)]=X(f)\exp(-j2\pi ft_0)$; \textbf{Freq shifting}: $\mathscr{F}[x(t)\exp(j2\pi f_0t)]=X(f-f_0)$; \textbf{Duality thm}: $\mathscr{F}[X(t)]=x(-f)$; \textbf{Modulation thm}: $\mathscr{F}[x(t)\cos(2\pi f_0t)]=\frac{1}{2}[X(f-f_0)+X(f+f_0)]$; \textbf{Time differentiation}: $\mathscr{F}[\frac{d^nx}{dt^n}]=(j2\pi f)^nX(f)$; \textbf{Freq diff}: $(-jt)^nx(t)\leftrightarrow\frac{d^nX}{df^n}$; \textbf{Time integration}: $\int_{-\infty}^tx(\tau)\,d\tau\leftrightarrow\frac{1}{2\pi f}X(f)+\frac{1}{2}X(0)\delta(f)$; \textbf{Time convolution}: $x_1(t)*x_2(t)\leftrightarrow X_1(f)X_2(f)$; \textbf{Multiplication}: $x_1(t)x_2(t)\leftrightarrow X_1(f)*X_2(f)$\\
    \textbf{Sampling thry}: ideal instantaneous sampled waveform of $x(t)$: $x_{\delta}(t)=\sum_{n=-\infty}^{\infty}x(nT_s)\delta(t-nT_s)$, where $T_s=\frac{1}{f_s}$ -- sampling interval; lowpass sgnl with bandwidth $W$ can be described by instantaneous sample values with sampling frequency $f_s>2W$ (Nyquist freq) and reconstructed by from the sampled waveform by passing it through an ideal LP filter with bandwidth $B$, where $W<B<f_s-W$; bandpass sgnl with bandwidth $W$ \& upper freq limit $f_u$ can be sampled at freq $f_s=2f_u/m$, where $m=\lfloor f_u/W\rfloor$, all higher sampling rates are not necessarily usable unless $f_s>2f_u$\\
    \textbf{Hilbert transform (HT)}: $\hat{x}(t)=h(t)*x(t)=\mathscr{F}^{-1}[H(f)X(f)]$, where $h(t)=\frac{1}{\pi t}$, \textbf{Freq response func}: $H(f)=-j\sgn(f)$; \textbf{HT properties}: energy of a sgnl and its HT are equal, $\abs{\hat{X}(f)}^2=\abs{X(f)}^2$; a sgnl and its HT are orthogonal, $\int_{-\infty}^{\infty}x(t)\hat{x}(t)\,dt=0$ (energy sgnl); if sgnl $c(t)$ \& $m(t)$ have no overlapping spectra, where $m(t)$ -- LP and $c(t)$ is HP, then $\widehat{m(t)c(t)}=m(t)\hat{c}(t)$\\
    \textbf{Analytic sgnl}: $x_p(t)=x(t)+j\hat{x}(t)$; \textbf{Envelope}: $\abs{x_p(t)}=\sqrt{x^2(t)+\hat{x}^2(t)}$; \textbf{Spectrum}: $X_p(f)=X(f)[1+\sgn(f)]$, removing negative spectrum, doubling positive spectrum\\
    \textbf{Complex Envelope $\tilde{x}(t)$}: $x_p(t)=\tilde{x}(t)e^{j2\pi f_0t}$, where $f_0$ -- reference freq\\
    \textbf{Inphase component}: $x_R(t)=\re[\tilde{x}(t)]$; \textbf{Quadrature component}: $x_I(t)=\im[\tilde{x}(t)]$\\

    \textbf{概率与统计}\\
    \textbf{联合概率(Joint probability)}:$P(AB)=P(A\cap B)=P(B)P(A\vert B)=P(A)P(B\vert A)$\\
    A\&B\textbf{独立(Independent)}$\leftrightarrow P(AB)=P(A)P(B)\leftrightarrow P(A\vert B)=P(A),=P(B\vert A)=P(B)$\\
    \textbf{全概率公式(Law of total prob)}:若$A_j$互斥(exclusive)$\forall j=1,\cdots n$,$\cup_{j=1}^nA_j=S$,则$P_B=\sum_{j=1}^nP(A_jB)=\sum_{j=1}^nP(B\vert A_j)P(A_j)$\\
    \textbf{贝叶斯定理(Bayes' Th)}:$P(A_i\vert B)=\frac{P(A_iB)}{P(B)}=\frac{P(B\vert A_i)P(A_i)}{\sum_{j=1}^nP(B\vert A_j)P(A_j)}$\\
    \textbf{随机变量(Random variable, r.v.)}:样本空间$S$至实数集的映射,$X(\cdot):A\subset S\rightarrow x\in R$\\
    \textbf{累计分布函数(Cumulative distribution func, CDF)}:$F_X(x)=P(X\leq x)$;$0\leq F_X(x)\leq 1$,$F_X(-\infty)=0$,$F_X(+\infty)=1$;非递减;$P(x_1<x\leq x_2)=F_X(x_2)-F_X(x_1)$\\
    \textbf{概率质量函数(Prob mass func, PMF)}:\\
    \textbf{概率密度函数(Prob density func)}:$f_X(x)=\frac{\mathrm{d}F_X(x)}{\mathrm{d}x}$,$F_X(x)=\int_{-\infty}^xF_X(\xi)\,\mathrm{d}\xi$;$f_X(x)\leq 0$;$\int_{-\infty}^{+\infty}f_X(x)\,\mathrm{d}x=1$;$P(x_1<X\leq x_2)=\int_{x_1}^{x_2}f_X(x)\,\mathrm{d}x$\\
    \textbf{r.v.的传递}:对$Y=g(X)$,若$g$单调,$f_Y(y)=f_X(x)\abs{\frac{\mathrm{d}x}{\mathrm{d}y}}_{x=g^{-1}(y)}$,否则$f_Y(y)=\sum_{i=1}^Nf_X(x)\abs{\frac{\mathrm{d}x_1}{\mathrm{d}y}}_{x_i=g^{-1}(y)}$\\
    \textbf{统计平均(Statistical ave)}:$E[X]=\bar{X}=m_X=\sum_{i=1}x_ip_X(x_i)$(离散),$=\int_{-\infty}^{+\infty}xf_X(x)\,\mathrm{d}x$(连续);对$Y=g(X)$,$E[Y]=\int_{-\infty}^{+\infty}yf_Y(y)\,\mathrm{d}y=\int_{-\infty}^{+\infty}g(x)f_X(x)\,\mathrm{d}x$\\
    \textbf{$n$阶矩(Moment)}:$E[X^n]=\int_{-\infty}^{+\infty}x^nf_X(x)\,\mathrm{d}x$;\textbf{$n$阶中心矩}:$E[(X-E[X])^n]=\int_{-\infty}^{+\infty}(x-E[X])^2f_X(x)\,\mathrm{d}x$;\textbf{方差}:即二阶中心矩,$\sigma_X^2=\mathrm{var}[x]=E[(X-E[X])^2]=E[X^2]-E^2[X]$,其中$\sigma_X$-\textbf{标准差}\\
    \textbf{伯努利分布(0-1分布,Bernoulli dist)}:若$X\sim b(p)$,$p_X(x)=\left\{\begin{array}{ll}
        1-p,&x=0\\
        p,&x=1
    \end{array}\right.=p^x(1-p)^x$,$E[X]=p$,$\sigma_X^2=p(1-p)$\\
    \textbf{二项式分布(Binomial dist)}:若$X_i\sim b(p)\forall i=1\cdots n$,对$Y=\sum_i^nX_i$,$p_Y(k)=C_k^np^k(1-p)^{n-k}$,其中组合数$C_k^n=\frac{n!}{k!(n-k)!}$\\
    \textbf{均匀分布(Uniform dist)}:若$X\sim U(a,b)$,$f_X(x)=\left\{\begin{array}{ll}
        \frac{1}{b-a},&a\leq x\leq b\\
        0,\text{otherwise}
    \end{array}\right.$,$E[X]=\frac{a+b}{2}$,$\sigma_X^2=\frac{(b-a)^2}{12}$\\
    \textbf{正态分布(高斯分布, Gaussian dist)}:若$X\sim N(\mu,\sigma^2)$,$f_X(x)=\frac{1}{\sqrt{2\pi\sigma^2}}\exp[-\frac{1}{2\sigma^2}(x-\mu)^2]$\\
    \textbf{$Q$函数}:$Q(x)=P(X\leq x)=\int_x^{+\infty}\frac{1}{\sqrt{2\pi}}\exp(-\frac{u^2}{2})\,\mathrm{d}u$,对$x\leq 0$,$Q(x)=\frac{1}{\pi}\int_0^{\pi/2}\exp(-\frac{x^2}{2\sin^2\theta})\,\mathrm{d}\theta\leq\frac{1}{2}\exp(-\frac{x^2}{2})$,其中$X\sim N(0,1)$;若$X\sim N(\mu,\sigma^2)$,$P(X>x)=Q(\frac{x-\mu}{\sigma})$\\
    \textbf{联合分布}:\textbf{联合分布函数(Joint dist func)}:$F_{XY}(x,y)=P(X\leq x,Y\leq y)$;\textbf{联合概率密度函数(Joint PDF)}:$f_{XY}(x,y)=\frac{\partial^2F_{XY}(x,y)}{\partial x\partial y}$;$\int_{-\infty}^{+\infty}\int_{-\infty}^{+\infty}f_{XY}(x,y)\,\mathrm{d}x\,\mathrm{d}y=1$;$P(x_1<X\leq x_2,y_1<Y\leq y_2)=\int_{y_1}^{y_2}\int_{x_1}^{x_2}f_{XY}(x,y)\,\mathrm{d}x\,\mathrm{d}y$;\textbf{边缘分布(Marginal dist)}:$F_X(x)=P(X\leq x,-\infty)$,$F_Y(y)=P(-\infty<X<+\infty,Y\leq y)$;\textbf{边缘密度函数(Marginal density)}:$f_X(x)=\int_{-\infty}^{+\infty}f_{XY}(x,y)\,\mathrm{d}y$,$f_Y(y)=\int_{-\infty}^{+\infty}f_{XY}(x,y)\,\mathrm{d}x$;$X$\&$Y$\textbf{独立}$\leftrightarrow F_{XY}(x,y)=F_X(x)F_Y(y)\leftrightarrow f_{XY}(x,y)=f_X(x)f_Y(y)\leftrightarrow f_{X\vert Y}(x\vert y)=f(x)$,$f_{Y\vert X}(y\vert x)=f_Y(y)$\\
    \textbf{r.v.线性组合}:对$Y=\sum_{i=1}^Na_iX_i$,$E[Y]=\sum_{i=1}^Na_iE[X_i]$,$\sigma_Y^2=\sum_{i=1}^Na_i^2\sigma_{X_i}$;对$Y=X+Y$,$X$\&$Y$独立,有$f_Z(z)=f_X*f_Y(z)=\int_{-\infty}^{+\infty}f_X(z-u)f_Y(u)\,\mathrm{d}u$\\
    \textbf{相关系数};\textbf{协方差(Covariance)}:$\sigma_{XY}=E[(X-E[X])(Y-E[Y])]=E[XY]-E[X]E[Y]$;\textbf{相关系数(Correlation coefficient)}:$\rho_{XY}=\frac{\sigma_{XY}}{\sigma_X\sigma_Y}$;$X$\&$Y$\textbf{不相关(Uncorrelated)}$\leftrightarrow\rho_{XY}$;若$E[XY]=E[X]E[Y]$,$X$\&$Y$不相关;若$X$\&$Y$独立,则不相关(逆命题假)\\
    \textbf{联合高斯分布}:$f_{X_1,\cdots,X_n}(x_1,\cdots,x_n)=[(2\pi)^n\det C]^{-1/2}\exp[-\frac{1}{2}[\bm{x}-\bm{\mu}]^TC^{-1}(\bm{x}-\bm{\mu})]$,其中$x=[x_1,\cdots,x_n]^T$,$\bm{\mu}=[\mu_1,\cdots,\mu_n]^T$,$C_{ij}=\sigma_{X_iX_j}$;对满足联合高斯分布的$X_1$,$X_2$,$C=\left[\begin{smallmatrix}
        \sigma_1^2&\sigma_{X_1X_2}\\
        \sigma_{X_1X_2}&\sigma_2^2
    \end{smallmatrix}\right]$,$f_{X_1X_2}(x_1,x_2)=\frac{1}{2\pi\sigma_1\sigma_2\sqrt{1-\rho^2}}\exp\{-\frac{1}{2(1-\rho^2)}[\frac{(x_1-\mu_1)^2}{\sigma_1^2}-\frac{2\rho(x_1-\mu_1)(x_2-\mu_2)}{\sigma_1\sigma_2}+\frac{(x_2-\mu_2)^2}{\sigma_2^2}]\}$;若$X_1$\&$X_2$不相关,$f_{X_1X_2}(x_1,x_2)=f_{X_1}(x_1)f_{X_2}(x_2)$\\
    若$x_1,\cdots,x_n$满足高斯联合分布,其任意子集满足高斯联合分布,其任意线性组合$\sim N(\sum_i^n\mu_i,\sigma_?)$\\
    \textbf{大数定律}:若$X_1,\cdots,X_n$等均值,方差且不相关,$Y=\frac{1}{n}\sum_{i=1}^nX_i$,则$\lim_{n\rightarrow\infty}P(\abs{Y-m_X}\leq\epsilon)=0,\forall\epsilon>0$,样本均值收敛至期望\\
    \textbf{中心极限定理(Central limit Th)}:若$X_1,\cdots,X_n$i.i.d.(\textbf{独立同分布}分布相同但相互独立),$Y=\frac{1}{n}\sum_{i=1}^nX_i$收敛至$N(m_X,\frac{\sigma_X^2}{n})$\\
    \textbf{瑞利分布}:$f_X(x)=\left\{\begin{array}{ll}
        \frac{x}{\alpha^2}\exp(-\frac{x^2}{2\alpha^2}),&x\geq 0\\
        0,&x<0
    \end{array}\right.$
\end{multicols}
\end{document}