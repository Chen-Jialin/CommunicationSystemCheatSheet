% !TEX program = pdflatex
% Introduction to Communication System Cheat Sheet
\documentclass[UTF8,a4paper,10pt]{article}
\usepackage[UTF8,scheme=plain,linespread=.1]{ctex}
\usepackage[margin=.1in]{geometry}
\usepackage{multicol}
\setlength{\columnseprule}{1pt}
\usepackage{amsmath,amssymb,mathrsfs,bm}
\allowdisplaybreaks[4]
\providecommand{\abs}[1]{\left\lvert#1\right\rvert}
\providecommand{\re}{\,\text{Re}\,}
\providecommand{\im}{\,\text{Im}\,}
\providecommand{\sgn}{\,\text{sgn}\,}
\providecommand{\sinc}{\,\text{sinc}\,}
\providecommand{\det}{\,\text{det}\,}
\usepackage{ulem}
\begin{document}
\scriptsize
\begin{multicols}{2}
    \noindent\textbf{模拟(analog)通信系统架构}:信源(source)$\rightarrow$\uline{调制器(modulator)}[发送器(transmitter)]$\rightarrow$信道(channel)$\rightarrow$\uline{探测器(detector)}[接收器(receiver)]$\rightarrow$用户(user);\textbf{数码(digital)通信系统架构}:信源$\rightarrow$\uline{A/D转换器(converter)$\rightarrow$信源编码器(source encoder,压缩冗余)$\rightarrow$信道编码器(channel encoder,增加冗余以便纠错)$\rightarrow$调制器}[发送器]$\rightarrow$信道$\rightarrow$\uline{探测器$\rightarrow$信道解码器(ch decoder)$\rightarrow$信源解码器(src decoder)$\rightarrow$D/A转换器}[接收器]$\rightarrow$用户\\
    \textbf{信号分类}:\textbf{模拟}:时间$\backslash$函数值均连续,\textbf{时间离散(discrete-time)}:时间离散,值连续,\textbf{数码}:时间$\backslash$值均离散;\textbf{确知(deterministic)}:关于时间的确定函数,\textbf{随机(random)}:在给定时间随机取值;\textbf{周期(periodic)}:$x(t+T_0)=x(t)\forall t\in\mathbb{R}$,其中$T_0$-周期,\textbf{非周期(aperiodic)}\\
    \textbf{通信系统性能指标}:\textbf{可靠性(reliability)}:模拟用\textbf{信噪比(signal-noise ratio,SNR)},数码用\textbf{位错率(bit err rate)}表征,\textbf{效率(efficiency)}:Band eff$=\frac{\text{bit rate}}{\text{bandwidth}}$bit/s/Hz, Energy eff$=\frac{\text{bit energy}}{\text{noise power spectral density}}$\\
    \textbf{Energy of sgnl}: $E=\int_{t_1}^{t_2}\abs{x(t)}^2\,dt$ (J); \textbf{Energy sgnl}: $\int_{-\infty}^{\infty}\abs{x(t)}^2\,dt<\infty$;; \textbf{Average power of sgnl}: $P=\frac{1}{t_2-t_1}\int_{t_1}^{t_2}\abs{x(t)}^2\,dt$ (W); \textbf{Power sgnl}: $0<\lim_{T\rightarrow\infty}\int_{-T/2}^{T/2}\abs{x(t)}^2\,dt<\infty$\\
    \textbf{Unit impulse (/delta) func $\delta(t)$}: $\int_{-\infty}^{+\infty}x(t)\delta(t-t_0)\,dt=x(t_0)$ where $x(t)$ -- any func continuous at $t_0=0$ (\textbf{sifting property}); \textbf{Properties}: $\int_{t_1}^{t_2}\delta(t-t_0)\,dt=1$ for $t_1<t_0<t_2$, $\delta(t-t_0)=0$ for $t\neq t_0$, $\delta(at)=\frac{\delta(t)}{\abs{a}}$, $\int_{t_1}^{t_2}x(t)\delta^{(n)}(t-t_0)\,dt=(-1)^nx^{(n)}(t_0)$ for $t_1<t_0<t_2$\\
    \textbf{Unit step func}: $u(t)=\int_{-\infty}^t\delta(\lambda)\,d\lambda=\left\{\begin{array}{ll}
        1,&t>0\\
        0,&t<0
    \end{array}\right.$\\
    \textbf{Unit rectangular pulse func}: $\Pi(t)=u(t+\frac{1}{2})-u(t-\frac{1}{2})$\\
    \textbf{Fourier Series (complex exp form)}: $x(t)=\sum_{n=-\infty}^{\infty}X_n\exp(jn2\pi f_0t),t_0\leq t<t_0+T_0$, where $f_0=\frac{1}{T_0}$, $X_n=\frac{1}{T_0}\int_{t_0}^{t_0+T_0}x(t)\exp(-jn2\pi f_0t)\,dt$; \textbf{(trigonometric form)}: $x(t)=\sum_{n=0}^{\infty}[A_n\cos(n2\pi f_0t)+B_n\sin(n2\pi f_0t)]$, where $A_n=\frac{2}{T}\int_{t_0}^{t_0+T_0}x(t)\cos(n2\pi f_0t)\,dt$, $B_n=\frac{2}{T_0}\int_{t_0}^{t_0+T_0}x(t)\sin(n2\pi f_0t)\,dt$\\
    \textbf{Fourier transform (FT) of a continuous-time sgnl}: $X(f)=\mathscr{F}[x(t)]=\int_{-\infty}^{\infty}x(\lambda)e^{-j2\pi f\lambda}\,d\lambda$, if $x(t)$ is absolutely integrable, i.e., $\int_{-\infty}^{\infty}\abs{x(t)}\,dt<\infty$; \textbf{Inverse Fourier transform}: $x(t)=\mathscr{F}^{-1}[X(f)]=\int_{-\infty}^{\infty}X(f)\exp(j2\pi ft)\,df$\\
    \textbf{Spectrum}: $X(f)=\abs{X(f)}\exp[j\angle X(f)]$\\
    \textbf{Parseval's theorem}: $\int_{-\infty}^{\infty}x_1(t)x_2^*(t)\,dt=\int_{-\infty}^{+\infty}X_1(f)X_2^*(f)\,df$\\
    For energy sgnl, \textbf{Energy spectral density}: $G(f)=\abs{X(f)}^2$ (J/Hz), energy: $E=\int_{-\infty}^{\infty}\abs{x(t)}^2\,dt=\int_{-\infty}^{\infty}\abs{X(f)}^2\,df$; \textbf{Time-average autocorrelation func}: $\phi(\tau)=x(\tau)*x(-\tau)=\int_{-\infty}^{\infty}x(\lambda)x(\lambda+\tau)\,d\tau=\lim_{T\rightarrow\infty}\int_{-T}^Tx(\lambda)x(\lambda+\tau)\,d\tau$; \textbf{Relation b/w ESP \& autocor}: $G(f)=\mathscr{F}[\phi(t)]$\\
    For power sgnl: \textbf{Power spectral density (PSD) $S(f)$}: ???; \textbf{Power}: $P=\int_{-\infty}^{\infty}S(f)\,df=\langle x^2(t)\rangle$; \textbf{Time-average autocorrelation func}: $R(\tau)=\langle x(t)x(t+\tau)\rangle=\lim_{T\rightarrow\infty}\frac{1}{2T}\int_{-T}^Tx(t)x(t+\tau)\,dt$; \textbf{Relation between PSD \& autocor}: $S(\tau)=\mathscr{F}[R(\tau)]$\\
    \textbf{Properties of FT}: for $x_{1/2}(t)\leftrightarrow X_{1/2}(f)$, \textbf{Superposition}: $\mathscr{F}[a_1x_1(t)+a_2x_2(t)]=a_1X_1(f)+a_2X_2(f)$; \textbf{Scaling}: $\mathscr[X(at)]=\frac{1}{\abs{a}}X(\frac{f}{a})$; \textbf{Time shifting}: $\mathscr{F}[x(t-t_0)]=X(f)\exp(-j2\pi ft_0)$; \textbf{Freq shifting}: $\mathscr{F}[x(t)\exp(j2\pi f_0t)]=X(f-f_0)$; \textbf{Duality thm}: $\mathscr{F}[X(t)]=x(-f)$; \textbf{Modulation thm}: $\mathscr{F}[x(t)\cos(2\pi f_0t)]=\frac{1}{2}[X(f-f_0)+X(f+f_0)]$; \textbf{Time differentiation}: $\mathscr{F}[\frac{d^nx}{dt^n}]=(j2\pi f)^nX(f)$; \textbf{Freq diff}: $(-jt)^nx(t)\leftrightarrow\frac{d^nX}{df^n}$; \textbf{Time integration}: $\int_{-\infty}^tx(\tau)\,d\tau\leftrightarrow\frac{1}{2\pi f}X(f)+\frac{1}{2}X(0)\delta(f)$; \textbf{Time convolution}: $x_1(t)*x_2(t)\leftrightarrow X_1(f)X_2(f)$; \textbf{Multiplication}: $x_1(t)x_2(t)\leftrightarrow X_1(f)*X_2(f)$\\
    \textbf{Sampling thry}: ideal instantaneous sampled waveform of $x(t)$: $x_{\delta}(t)=\sum_{n=-\infty}^{\infty}x(nT_s)\delta(t-nT_s)$, where $T_s=\frac{1}{f_s}$ -- sampling interval; lowpass sgnl with bandwidth $W$ can be described by instantaneous sample values with sampling frequency $f_s>2W$ (Nyquist freq) and reconstructed by from the sampled waveform by passing it through an ideal LP filter with bandwidth $B$, where $W<B<f_s-W$; bandpass sgnl with bandwidth $W$ \& upper freq limit $f_u$ can be sampled at freq $f_s=2f_u/m$, where $m=\lfloor f_u/W\rfloor$, all higher sampling rates are not necessarily usable unless $f_s>2f_u$\\
    \textbf{Hilbert transform (HT)}: $\hat{x}(t)=h(t)*x(t)=\mathscr{F}^{-1}[H(f)X(f)]$, where $h(t)=\frac{1}{\pi t}$, \textbf{Freq response func}: $H(f)=-j\sgn(f)$; \textbf{HT properties}: energy of a sgnl and its HT are equal, $\abs{\hat{X}(f)}^2=\abs{X(f)}^2$; a sgnl and its HT are orthogonal, $\int_{-\infty}^{\infty}x(t)\hat{x}(t)\,dt=0$ (energy sgnl); if sgnl $c(t)$ \& $m(t)$ have no overlapping spectra, where $m(t)$ -- LP and $c(t)$ is HP, then $\widehat{m(t)c(t)}=m(t)\hat{c}(t)$\\
    \textbf{Analytic sgnl}: $x_p(t)=x(t)+j\hat{x}(t)$; \textbf{Envelope}: $\abs{x_p(t)}=\sqrt{x^2(t)+\hat{x}^2(t)}$; \textbf{Spectrum}: $X_p(f)=X(f)[1+\sgn(f)]$, removing negative spectrum, doubling positive spectrum\\
    \textbf{Complex Envelope $\tilde{x}(t)$}: $x_p(t)=\tilde{x}(t)e^{j2\pi f_0t}$, where $f_0$ -- reference freq\\
    \textbf{Inphase component}: $x_R(t)=\re[\tilde{x}(t)]$; \textbf{Quadrature component}: $x_I(t)=\im[\tilde{x}(t)]$\\
    \textbf{概率与统计}\hrulefill\\
    \textbf{联合概率(joint probability)}:$P(AB)=P(A\cap B)=P(B)P(A\vert B)=P(A)P(B\vert A)$\\
    A\&B\textbf{独立(independent)}$\Leftrightarrow P(AB)=P(A)P(B)\Leftrightarrow P(A\vert B)=P(A)$,$P(B\vert A)=P(B)$\\
    \textbf{全概率公式(law of total prob)}:若$A_j$互斥(exclusive)$\forall j=1,\cdots,n$,$\cup_{j=1}^nA_j=S$,则$P(B)=\sum_{j=1}^nP(A_jB)=\sum_{j=1}^nP(B\vert A_j)P(A_j)$\\
    \textbf{贝叶斯定理(Bayes' thm)}:$P(A_i\vert B)=\frac{P(A_iB)}{P(B)}=\frac{P(B\vert A_i)P(A_i)}{\sum_{j=1}^nP(B\vert A_j)P(A_j)}$\\
    \textbf{随机变量(random variable,r.v.)}:样本空间$S$至实数集的映射,$X(\cdot):A\subset S\rightarrow x\in R$\\
    \textbf{累计分布函数(cumulative distribution func,CDF)}:$F_X(x)=P(X\leq x)$;$0\leq F_X(x)\leq 1$,$F_X(-\infty)=0$,$F_X(+\infty)=1$;非递减;$P(x_1<x\leq x_2)=F_X(x_2)-F_X(x_1)$\\
    对离散r.v.,\textbf{概率质量函数(prob mass func,PMF)}:$p_X(x)=P(X=x)$\\
    对连续r.v.,\textbf{概率密度函数(prob density func,PDF)}:$f_X(x)=\frac{\mathrm{d}F_X(x)}{\mathrm{d}x}$,$F_X(x)=\int_{-\infty}^xf_X(\xi)\,\mathrm{d}\xi$;$f_X(x)\geq 0$;$\int_{-\infty}^{+\infty}f_X(x)\,\mathrm{d}x=1$;$P(x_1<X\leq x_2)=\int_{x_1}^{x_2}f_X(x)\,\mathrm{d}x$;对$Y=g(X)$,若$g$单调,$f_Y(y)=f_X(x)\abs{\frac{\mathrm{d}x}{\mathrm{d}y}}_{x=g^{-1}(y)}$,否则$f_Y(y)=\sum_{i=1}f_X(x)\abs{\frac{\mathrm{d}x_i}{\mathrm{d}y}}_{x_i=g^{-1}(y)}$\\
    \textbf{统计平均(statistical average)}:$E[X]=\bar{X}=m_X=\sum_{i=1}x_ip_X(x_i)$(离散),$E[X]=\int_{-\infty}^{+\infty}xf_X(x)\,\mathrm{d}x$(连续);对$Y=g(X)$,$E[Y]=\int_{-\infty}^{+\infty}yf_Y(y)\,\mathrm{d}y=\int_{-\infty}^{+\infty}g(x)f_X(x)\,\mathrm{d}x$\\
    \textbf{$n$阶矩($n$th moment)}:$E[X^n]=\int_{-\infty}^{+\infty}x^nf_X(x)\,\mathrm{d}x$;\textbf{$n$阶中心矩($n$th central moment)}:$E[(X-E[X])^n]=\int_{-\infty}^{+\infty}(x-E[X])^2f_X(x)\,\mathrm{d}x$;\textbf{方差}:即二阶中心矩,$\mathrm{var}[X]=\sigma_X^2=E[(X-E[X])^2]=E[X^2]-E^2[X]$,其中$\sigma_X$-\textbf{标准差}\\
    \textbf{伯努利分布(0-1分布,Bernoulli dist)}:若$X\sim b(p)$,$p_X(x)=\left\{\begin{array}{ll}
        1-p,&x=0\\
        p,&x=1
    \end{array}\right.=p^x(1-p)^x$,$E[X]=p$,$\sigma_X^2=p(1-p)$\\
    \textbf{二项式分布(binomial dist)}:若$X_i\sim b(p)\forall i=1,\cdots,n$,$Y=\sum_{i=1}^nX_i\sim B(n,p)$,$p_Y(k)=C_k^np^k(1-p)^{n-k}$,其中组合数$C_k^n=\frac{n!}{k!(n-k)!}$,$E[Y]=np$,$\sigma_Y^2=np(1-p)$\\
    \textbf{均匀分布(uniform dist)}:若$X\sim U(a,b)$,$f_X(x)=\left\{\begin{array}{ll}
        \frac{1}{b-a},&a\leq x\leq b\\
        0,&\text{otherwise}
    \end{array}\right.$,$E[X]=\frac{a+b}{2}$,$\sigma_X^2=\frac{(b-a)^2}{12}$\\
    \textbf{正态(/高斯)分布(normal/Gaussian dist)}:若$X\sim N(E[X]=\mu,\mathrm{var}[X]=\sigma^2)$,$f_X(x)=\frac{1}{\sqrt{2\pi\sigma^2}}\exp[-\frac{1}{2\sigma^2}(x-\mu)^2]$\\
    \textbf{$Q$函数}:$Q(x)=P(X\geq x)=\int_x^{+\infty}\frac{1}{\sqrt{2\pi}}\exp(-\frac{\xi^2}{2})\,\mathrm{d}\xi$,其中$X\sim N(0,1)$;对$x\geq 0$,$Q(x)=\frac{1}{\pi}\int_0^{\pi/2}\exp(-\frac{x^2}{2\sin^2\theta})\,\mathrm{d}\theta\leq\frac{1}{2}\exp(-\frac{x^2}{2})$;若$X\sim N(\mu,\sigma^2)$,$P(X>x)=Q(\frac{x-\mu}{\sigma})$\\
    \textbf{联合分布(joint dist)}:$F_{XY}(x,y)=P(X\leq x,Y\leq y)$;\textbf{联合概率密度函数(joint PDF)}:$f_{XY}(x,y)=\frac{\partial^2F_{XY}(x,y)}{\partial x\partial y}$;$\int_{-\infty}^{+\infty}\int_{-\infty}^{+\infty}f_{XY}(x,y)\,\mathrm{d}x\,\mathrm{d}y=1$;$P(x_1<X\leq x_2,y_1<Y\leq y_2)=\int_{y_1}^{y_2}\int_{x_1}^{x_2}f_{XY}(x,y)\,\mathrm{d}x\,\mathrm{d}y$;\textbf{边缘分布(marginal dist)}:$F_X(x)=P(X\leq x,-\infty<y<+\infty)$,$F_Y(y)=P(-\infty<X<+\infty,Y\leq y)$;\textbf{边缘密度函数(marginal density)}:$f_X(x)=\int_{-\infty}^{+\infty}f_{XY}(x,y)\,\mathrm{d}y$,$f_Y(y)=\int_{-\infty}^{+\infty}f_{XY}(x,y)\,\mathrm{d}x$;$X$\&$Y$\textbf{独立}$\Leftrightarrow F_{XY}(x,y)=F_X(x)F_Y(y)\Leftrightarrow f_{XY}(x,y)=f_X(x)f_Y(y)\Leftrightarrow f_{X\vert Y}(x\vert y)=f(x)$,$f_{Y\vert X}(y\vert x)=f_Y(y)$\\
    \textbf{r.v.的线性组合}:对$Y=\sum_{i=1}a_iX_i$,$E[Y]=\sum_{i=1}a_iE[X_i]$,$\sigma_Y^2=\sum_{i=1}a_i^2\sigma_{X_i}^2$;对$Z=X+Y$,$X$\&$Y$独立,有$f_Z(z)=f_X(z)*f_Y(z)=\int_{-\infty}^{+\infty}f_X(z-u)f_Y(u)\,\mathrm{d}u$\\
    \textbf{相关(correlation)函数}:$R_{XY}=E[XY]=\int_{-\infty}^{+\infty}\int_{-\infty}^{+\infty}xyf_{XY}(x,y)\,\mathrm{d}x\,\mathrm{d}y$;\textbf{协方差(covariance)}:即$\sigma_{XY}=E[(X-E[X])(Y-E[Y])]=E[XY]-E[X]E[Y]$;\textbf{相关系数(corr coefficient)}:$\rho_{XY}=\frac{\sigma_{XY}}{\sigma_X\sigma_Y}$;$X$\&$Y$\textbf{不相关(uncorrelated)}$\Leftrightarrow\rho_{XY}=0$;独立$\Rightarrow$不相关,逆命题假\\
    \textbf{联合高斯分布}:$f_{X_1,\cdots,X_n}(x_1,\cdots,x_n)=[(2\pi)^n\det C]^{-1/2}\exp[-\frac{1}{2}[\bm{x}-\bm{\mu}]^TC^{-1}(\bm{x}-\bm{\mu})]$,其中$x=[x_1,\cdots,x_n]^T$,$\bm{\mu}=[\mu_1,\cdots,\mu_n]^T$,$C_{ij}=E[(X_i-\mu_i)(X_j-\mu_j)]$;若$X_1$,$X_2$满足联合高斯分布,$C=\left[\begin{smallmatrix}
        \sigma_1^2&\sigma_{X_1X_2}\\
        \sigma_{X_1X_2}&\sigma_2^2
    \end{smallmatrix}\right]$,$f_{X_1X_2}(x_1,x_2)=\frac{1}{2\pi\sigma_1\sigma_2\sqrt{1-\rho^2}}\exp\{-\frac{1}{2(1-\rho^2)}[\frac{(x_1-\mu_1)^2}{\sigma_1^2}-\frac{2\rho(x_1-\mu_1)(x_2-\mu_2)}{\sigma_1\sigma_2}+\frac{(x_2-\mu_2)^2}{\sigma_2^2}]\}$;若$X_1$\&$X_2$不相关,$f_{X_1X_2}(x_1,x_2)=\frac{1}{2\pi\sigma_1\sigma_2}\exp\{-\frac{1}{2}[\frac{(x_1-\mu_1)^2}{\sigma_1^2}+\frac{(x_2-\mu_2)^2}{\sigma_2^2}]\}=f_{X_1}(x_1)f_{X_2}(x_2)$;若$X_1,\cdots,X_n$满足高斯联合分布,其任意子集满足高斯联合分布,任意线性组合满足高斯分布\\
    \textbf{大数定律(law of large number)}:若$X_1,\cdots,X_n$等均值,等方差且不相关,$Y=\frac{1}{n}\sum_{i=1}^nX_i$,则$\lim_{n\rightarrow\infty}P(\abs{Y-m_X}\leq\epsilon)=0\forall\epsilon>0$,样本均值收敛至期望\\
    \textbf{中心极限定理(central limit thm)}:若$X_1,\cdots,X_n$\textbf{独立同分布}(i.i.d.:分布相同但相互独立),$Y=\frac{1}{n}\sum_{i=1}^nX_i$收敛至$N(m_X,\sigma_X^2/n)$\\
    \textbf{瑞利分布(Rayleigh dist)}:$f_X(x)=\left\{\begin{array}{ll}
        \frac{x}{\alpha^2}\exp(-\frac{x^2}{2\alpha^2}),&x\geq 0\\
        0,&x<0
    \end{array}\right.$;若$X$,$Y$i.i.d.$\sim N(0,\sigma^2)$,$R=\sqrt{X^2+Y^2}\sim$瑞利分布$(\alpha=\sigma)$,$\Theta=\arctan\frac{Y}{X}$,证:$X=R\cos\Theta$,$Y=R\sin\Theta$,$\frac{\partial(x,y)}{\partial(r,\theta)}=\left\lvert\begin{smallmatrix}
        \cos\theta&-r\sin\theta\\
        \sin\theta&r\cos\theta
    \end{smallmatrix}\right\rvert=r$,$f_{R\Theta}(r,\theta)=\frac{\partial(x,y)}{\partial(r,\theta)}\frac{1}{2\pi\sigma^2}\exp[-(x^2+y^2)/2\sigma^2]=\frac{r\exp(-r^2/2\sigma^2)}{2\pi\sigma^2}$,$0\leq r<+\infty$,$0\leq\theta<\infty$\\
    \rule{\columnwidth}{.5pt}\\
    \textbf{随机过程(random process)}:随机变量关于时间的演化,$X(t,\zeta)$,其中$t\in\mathbb{R}$-时间,$\zeta\in S$-事件,$X(t_0,\zeta)$-随机变量,$X(t,\zeta_0)$-\textbf{抽样函数(sample func)},$X(t_0,\zeta_0)$-函数值,$X(t)=\{X(t,\zeta_1),\cdots,X(t,\zeta_n)\}$-穷尽所有可能$\zeta$所得抽样函数的集合\\
    \textbf{随机过程的统计}:对$\{X(t_1),\cdots,X(t_n)\}$,\textbf{联合PDF}$f_{X(t_1),\cdots,X(t_n)}(x_1,\cdots,x_n)=f_{X_1,\cdots,X_n}(x_1,\cdots,x_n)$\\
    \textbf{一阶统计量(1st order statistics)}:$X(t)$在$t$的PDF:$f_{X(t)}(x)$;\textbf{均值}:$E[X(t)]=\int_{-\infty}^{+\infty}xf_{X(t)}(x)\,\mathrm{d}x$,\textbf{方差}:$\sigma_X^2(t)=E[(X(t)-E[X(t)])^2]$\\
    \textbf{二阶统计量}:\textbf{自相关函数(autocorr func)}:$R_X(t_1,t_2)=E[X(t_1)X(t_2)]=\int_{-\infty}^{+\infty}\int_{-\infty}^{+\infty}x_2x_2f_{X(t_1)X(t_2)}(x_1,x_2)\,\mathrm{d}x_1\,\mathrm{d}x_2$,\textbf{自协方差(autocov)}:$\mu_X(t_1,t_2)=E[(X(t_1)-E[X(t_1)])(X(t_2)-E[X(t_2)])]=\int_{-\infty}^{+\infty}\int_{-\infty}^{+\infty}(x_1-E[X(t_1)])(x_2-E[X(t_2)])f_{X(t_1)X(t_2)}(x_1,x_2)\,\mathrm{d}x_1\,\mathrm{d}x_2=R_X(t_1,t_2)-E[X(t_1)]E[X(t_2)]$;与方差关系:$\sigma_X^2(t)=\mu_X(t,t)$\\
    \textbf{(严/狭义)平稳随机过程((strictly) stationary process)}:$f_{X(t_1),\cdots,X(t_n)}(x_1,\cdots,x_n)=f_{X(t_1+\tau),\cdots,X(t_n+\tau)}(x_1,\cdots,x_n)\forall\{n\},\tau$;必要条件:一阶统计量$E[X(t)]$,$\sigma_X^2(t)$不依赖$t$,二阶统计量$R_X(t_1,t_2)=R_X(t_2-t_1)$仅依赖时间间隔\\
    \textbf{宽(/广义)平稳随机(Wide-sense stationary, WSS)过程}:一阶统计量$E[X(t)]$,$\sigma_X^2(t)$不依赖$t$,二阶统计量$R_X(t_1,t_2)=R_X(t_2-t_1)$仅依赖时间间隔(充要条件);严平稳随机过程必为宽平稳随机过程,逆命题假;多数场景下平稳指代WSS\\
    \textbf{统计/系综平均(ensemble/statistical ave)}:$E[X(t)]$,$R_X(t,t+\tau)$,对样本平均,得关于$t$的函数;\textbf{时间平均(time ave)}:$\langle X(t)\rangle=\lim_{T\rightarrow\infty}\frac{1}{2T}\int_{-T}^Tx(t)\,\mathrm{d}t$,$\langle X(t)X(t+\tau)\rangle=\lim_{T\rightarrow\infty}\frac{1}{2T}\int_{-T}^Tx(t)x(t+\tau)\,\mathrm{d}t$;对周期函数,时间平均可在一个周期内进行;\textbf{各态历经(ergodic)}:统计平均等于时间平均,$E[X(t)]=\langle X(t)\rangle$,$R_X(t,t+\tau)=\langle X(t)X(t+\tau)\rangle$;各态历经$\Rightarrow$WSS,逆命题假;一般认为,信号-WSS,噪音严平稳,时变信道ergodic\\
    \textbf{功率谱密度(power spectral density,PSD)}:对确知信号,$S_X(f)=\lim_{T\rightarrow\infty}\frac{1}{T}\abs{X_T(f)}^2$;对随机过程,$S_X(f)=\lim_{T\rightarrow\infty}\frac{1}{T}E[\abs{X_T(f)}^2]$,其中$X_T(t)=\Pi(\frac{t}{T})X(t)$,$X_T(f)=\mathscr{F}[X_T(t)]$\\
    \textbf{维纳-辛钦定理(Winner-Khinchine thm)}:若WSS,$S_X(f)=\mathscr{F}[R_X(\tau)]$;\textbf{平均功率}$P=R_X(0)=E[X^2(t)]=\int_{-\infty}^{+\infty}S_X(f)\,\mathrm{d}f$;$S_X(0)=\int_{-\infty}^{+\infty}R_X(\tau)\,\mathrm{d}\tau$;$\abs{R(\tau)}\leq R(0)$,证:$E[(X(t)\pm X(t+\tau))^2]=2R_X(0)\pm 2R_X(\tau)\geq 0$即得;$S_X(f)\geq 0\forall f$;$\because R(-\tau)=R(\tau)$,$\therefore S_X(f)=S_X(-f)$;若$X(t)$不含周期性分量,$\lim_{\abs{\tau}\rightarrow\infty}R(\tau)=E^2[X(t)]$;若ergodic,$S_X(f)=\mathscr{F}[\langle X(t)X(t+\tau)\rangle]$\\
    对WSS且联合WWS($f_{X(t)Y(t+\tau)}(x,y)$不依赖$t$)的$X$,$Y$,\textbf{互相关(cross corr)函数}:$R_{XY}(\tau)=E[X(t)Y(t+\tau)]$;$X$\&$Y$\textbf{正交(orthogonal)}$\Leftrightarrow R_{XY}(\tau)=0\forall\tau$;$R_{XY}(\tau)=R_{YX}(-\tau)$;$n(t)=X(t)+Y(t)$的功率:$E[n^2(t)]=E[X^2(t)]+2E[X(t)Y(t)]+E[Y^2(t)]=E[X^2(t)]+2R_{XY}(0)+E[Y^2(t)]$;\textbf{交叉功率谱密度}:$S_{XY}(f)=\mathscr{F}[R_{XY}(\tau)]$\\
    随机过程经线性系统:输出$Y(t)=X(t)*h(t)$,$Y(f)=X(f)H(f)$,其中$X(t/f)$-输入,$h(t)/H(f)$-线性系统脉冲响应,\textbf{均值}$E[Y(t)]=\int_{-\infty}^{+\infty}h(\tau)E[X(t-\tau)]\,\mathrm{d}\tau$,若$X(t)$WSS,$E[Y(t)]=E[X(t)]\int_{-\infty}^{+\infty}h(\tau)\,\mathrm{d}\tau=E[X(t)]H(0)$;\textbf{自相关函数}$R_Y(t_1,t_2)=E[Y(t_1)Y(t_2)]=E[\int_{-\infty}^{+\infty}h(\tau_1)X(t-\tau_1)\,\mathrm{d}\tau_1\int_{-\infty}^{+\infty}h(\tau_2)X(t_2-\tau_2)\,\mathrm{d}\tau_2]=\int_{-\infty}^{+\infty}h(\tau_1)\,\tau_1\int_{-\infty}^{+\infty}h(\tau_2)E[X(t-\tau_1)X(t-\tau_2)]\,\mathrm{d}\tau_2$,若$X(t)$WSS,$R_Y(\tau)=\int_{-\infty}^{+\infty}\int_{-\infty}^{+\infty}h(\tau_1)h(\tau_2)R_X(\tau-\tau_1+\tau_2)\,\mathrm{d}\tau_1\,\mathrm{d}\tau_2=\int_{-\infty}^{+\infty}h(\tau_2)[h(\tau+\tau_2)*R(\tau+\tau_2)]\,\mathrm{d}\tau_2=h(-\tau)*h(\tau)*R_X(\tau)$(利用了$x(-\tau)*y(\tau)=\int_{-\infty}^{+\infty}x(\tau_1)h(\tau+t_1)\,\mathrm{d}t_1$);若输入WSS,输出WSS;\textbf{PSD}:$S_Y(\tau)=\abs{H(f)}^2S_X(f)$\\
    \rule{\columnwidth}{.5pt}\\
    \textbf{高斯随机过程}:$f_{X(t_1),\cdots,X(t_n)}(x_1,\cdots,x_n)=[(2\pi)^n\det C]^{1/2}\exp[\frac{1}{2}(\bm{x}-\bm{m})^TC^{-1}(\bm{x}-\bm{m})]$;对高斯随机过程,WSS$\Leftrightarrow$严平稳;$X(t_1),\cdots,X(t_n)$不相关$\Rightarrow$独立;高斯随机过程经线性系统仍为高斯随机过程\\
    \textbf{噪声(noise)}:通常建模为均值为$0$的平稳高斯随机过程;\textbf{白噪声(white noise)}:PSD为常数[$S_w(f)=\frac{N_0}{2}$,其中$N_0=k_BT\approx$(常温下)$4.14\times 10^{-21}\mathrm{W}=-174\mathrm{dBm/Hz}$-单边PSD,$\frac{N_0}{2}$-双边…]的噪声;对白噪声,$R_w(\tau)=\frac{N_0}{2}\delta(\tau)$,任意两个不同时刻不相关\\
    \textbf{带限噪声(bandlimited noise)}:$S_n(f)=\frac{N_0}{2}\Pi(\frac{f}{B})$,总功率$N_0B$,$R_n(\tau)=\mathscr{F}[S_n(f)]=N_0B\sinc(2B\tau)$,取采样率为$2B$Hz,可得到不相关的信号\\
    \textbf{噪声等效带宽(noise equivalent bandwidth)}:响应函数最大值与真实滤波器相等($H_0=\max{H(f)}$)且滤出噪音平均功率$P_n$相等的理想滤波器的带宽$B_N=\frac{\int_0^{+\infty}\abs{H(f)}^2\,\mathrm{d}f}{H_0}$,证:真实滤波器$H(f)$滤白噪声得平均功率$P_n=\int_{-\infty}^{+\infty}\frac{N_0}{2}\abs{H(f)}^2\,\mathrm{d}f$,$\because h(t)\in\mathbb{R}$,$\abs{H(-f)}=\abs{H(f)}$,$P_n=N_0\int_0^{+\infty}\abs{H(f)}^2\,\mathrm{d}f$,理想滤波器滤白噪声得平均功率$P_n=N_0H_0^2B_N$,$P_n=P_n$即得\\
    \textbf{窄带噪声(narrowband noise)}:$n(t)=n_c(t)\cos(2\pi f_0t+\theta)-n_s(t)\sin(2\pi f_0t+\theta)$,其中\textbf{同相分量(in-phase component)}$n_c(t)=\text{Lp}[2\cos(2\pi f_0t+\theta)\cdot n(t)]$,\textbf{正交分量(quadrature component)}$n_s(t)=\text{Lp}[-2\sin(2\pi f_0t+\theta)\cdot n(t)]$,$\theta$为任意角度(通常取$\theta=0$或$\theta\sim U[0,2\pi)$),$\text{Lp}[]$-低通滤波,证:$E\{[n(t)-[n_c(t)\cos(2\pi f_0t+\theta)-n_s(t)\sin(2\pi f_0t+\theta)]]^2\}=0$;若$n(t)$为高斯随机过程,$n_c(t)$\&$n_s(t)$为联合高斯随机过程;若$n(t)$平稳,$n_s(t)$\&$n_c(t)$联合平稳;\textbf{均值}:$E[n(t)]=E[n_c(t)]=E[n_s(t)]=0$;\textbf{方差}:$E[n^2(t)]=E[n_c^2(t)]=E[n_s^2(t)]$;\textbf{功率谱密度}:$S_{n_c}(f)=S_{n_s}(f)=\text{Lp}[S_n(f-f_0)+S_n(f+f_0)]$;\textbf{相关函数}:$R_{n_c}(\tau)=R_{n_s}(\tau)$;$R_n(0)=R_{n_c}(0)=R_{n_s}(0)$;\textbf{交叉功率谱密度}:$S_{n_cn_s}(f)=j\text{Lp}[S_n(f-f_0)-S_n(f+f+f_0)]$;若$n(t)$WSS,\textbf{交叉相关函数}奇,$R_{n_cn_s}(-\tau)=-R_{n_cn_s}(\tau)$,$R_{n_s}(0)=R_{n_c}(0)$;若$n(t)$在$f>0$处关于$f_0$对称,$\text{Lp}[S_n(f-f_0)-S_n(f+f_0)]=0$,\textbf{交叉相关函数}$R_{n_cn_s}(\tau)=0\forall\tau$,同向分量与正交分量不相关\\
    \textbf{窄带噪声}:$n(t)=R(t)\cos[2\pi f_0t+\theta+\phi(t)]$,其中\textbf{包络(envelope)}$R(t)=\sqrt{n_c^2(t)+n_s^2(t)}$遵循瑞利分布,$f_R(r)=\frac{r}{\sigma^2}\exp(-\frac{r^2}{2\sigma^2})$,$r\geq 0$,\textbf{相位(phase)}:$\phi(t)=\arctan\frac{n_s(t)}{n_c(t)}$满足均匀分布,$f_{\phi}(\varphi)=\frac{1}{2\pi}$,$0\leq\varphi<2\pi$\\
    % 含有窄带噪声的正弦信号
    \rule{\columnwidth}{.5pt}\\
    \textbf{模拟调制}\\
    \textbf{调制(modulation)}:将一个信号转化为另一信号以便在信道中传输;\textbf{调制的过程}:首先在发送器产生一载波,再按照需要传输的信息调节载波的某些特征,最后在接收器解调获得信息;\textbf{调制的原因}:频谱搬移,频分复用,增强抗噪声性能\\
    \textbf{模拟调制的种类}:\textbf{幅度调制(调幅,amplitude modul,AM)},是一种\textbf{线性调制(linear modul)},\textbf{角度调制(angle modul)},包括\textbf{频率调制(调频,freq modul,FM)}和\textbf{相位调制(调相,phase modul,PM)},是一种\textbf{非线性调制(non-linear modul)}\\
    \textbf{双边带抑制载波AM(double-sideband suppressed-carrier AM,DSB-SC)}.设\textbf{基带信号(baseband/modulating sgnl)}$m(t)$,载波$C(t)=A_c\cos(2\pi f_ct)$.\textbf{调制}:\textbf{调制信号(modulated sgnl)}$x_c(t)=m(t)C(t)=A_cm(t)\cos(2\pi f_ct)$,$X_c(f)=\frac{1}{2}A_cM(f+f_c)+\frac{1}{2}A_cM(f-f_c)$.\textbf{解调(demodulation)}:\textbf{相干解调(phase-coherent demod)}:用\textbf{锁相环(phase-lock loop,PLL)}产生一\textbf{本地振荡(local oscillator,LO)}$2\cos 2\pi f_ct$,与接收到的信号相乘,得$d(t)=x_c(t)\cdot 2\cos(2\pi f_ct)=A_cm(t)(1+\cos 4\pi f_ct)$,经低通滤波器$y_D(t)=A_cm(t)$.\textbf{优}:\textbf{功率效率}(传输功率占输入功率之比)$100\%$;\textbf{缺}:占用带宽($B=2W$,其中$W$-$m(t)$的带宽)多,解调难且易引入相位误差,从而造成畸变(假设存在时变相差$d(t)=x_c(t)\cdot 2\cos[2\pi f_ct+\theta(t)]=A_cm(t)[\cos\theta(t)+\cos(4\pi f_ct+\theta(t))]$,则解调后输出存在严重畸变$y_D(t)=m(t)\cos\theta(t)$)\\
    \textbf{有载波双边带AM(double-sideband large-carrier AM,DSB-LC,一般意义上的AM)}$=\text{DSB-SC}+\text{载波分量}$.\textbf{调制}:乘系数$\frac{a}{\abs{\min[m(t)]}}$并加$1$后与载波分量相乘(或乘系数并与载波分量相乘后再加一载波分量)$x_c(t)=A_c[1+am_n(t)]\cos(2\pi f_ct)$,其中$a\in(0,1]$-\textbf{调制指数(modul index)},$m_n(t)=\frac{m(t)}{\abs{\min[m(t)]}}\geq-1$,以使$[1+am_n(t)]\geq 0\forall t$.\textbf{解调}:可用相干解调,但复杂而没必要,一般用\textbf{包络检波(envelope demod)}(属于\textbf{非相干解调(noncoherent demod)}):使用$RC$电路做低通滤波,条件:$1+am_n(t)\geq 0$且$f_c\gg W$.\textbf{传输效率}:信号功率($P_s=\frac{1}{2}A_c^2a^2\langle m_n^2(t)\rangle$)在总传输功率($P_t=\langle x_c^2(t)\rangle=\langle A_c^2[1+am_n(t)]^2\cos^2(2\pi f_ct)\rangle=\frac{1}{2}A_c^2+\frac{1}{2}A_c^2a^2\langle m_n^2(t)\rangle$,其中利用了$m_n(t)$较$\cos(4\pi f_ct)$变化得慢很多及$E[m_n(t)]=0$的假设)中的占比,$\mu=\frac{P_s}{P_t}=\frac{a^2\langle m_n^2(t)\rangle}{a^2\langle m_n^2(t)\rangle+1}$;当$\abs{\min[m(t)]}=\abs{\max[m(t)]}$,$a=1$时,$\mu_{\max}=50\%$.\textbf{优}:包络检波易且廉;\textbf{缺}:功率效率较DSB-SC低,占用带宽($B=2W$)大\\
    \textbf{单边带(SSB)调制}:\textbf{方法1:边带滤波(sideband filtering)}:先产生一个DSB-SC信号,滤除一对(上/下)边带,\textbf{条件}:$m(t)$不包含过多低频分量,或使用理想滤波器;\textbf{方法2:移相(phase shift)}:先产生一个$m(t)$的正交信号$\hat{m}(t)$(做希尔伯特变换),$m(t)$和$\hat{m}(t)$分别乘$\cos 2\pi f_ct$和$\sin 2\pi f_ct$再相减/加得上/下边带(U/LSB-)SSB,$x_c(t)=\frac{1}{2}A_cm(t)\cos 2\pi f_ct\mp\frac{1}{2}A_c\hat{m}(t)\sin 2\pi f_ct$,\textbf{要求}:精确的移相.\textbf{解调}:\textbf{方法1:同步检波(synchronous detection)}:先乘一本地振荡$4\cos[2\pi f_ct+\theta(t)]$,得$x_c(t)\cdot 4\cos[2\pi f_ct+\theta(t)]=A_cm(t)\{\cos[\theta(t)]+\cos[4\pi f_ct+\theta(t)]\}\pm A_c\hat{m}(t)\{\sin[\theta(t)]-\sin[4\pi f_ct+\theta(t)]\}$,其中$\theta(t)$-时变相差,再经低通滤波得$y_D(t)=\text{Lp}[x_c(t)C(t)]=m(t)\cos\theta(t)\pm\hat{m}(t)\sin\theta(t)$,受相差干扰而易产生畸变;\textbf{方法2:载波重置(carrier reinsertion)}:先加一载波$K\cos 2\pi f_ct$,得$e(t)=[\frac{1}{2}A_cm(t)+K]\cos 2\pi f_ct\mp\frac{1}{2}A_c\hat{m}(t)\sin 2\pi f_ct$,再包络检波,得$y_D(t)=\sqrt{[\frac{1}{2}A_cm(t)+K]^2+[\frac{1}{2}A_c\hat{m}(t)]^2}$,当设定较大的$K$使$[\frac{1}{2}A_cm(t)+K]^2\gg[\frac{1}{2}A_c\hat{m}(t)]^2$,\textbf{要求}:加入大载波,与原调制载波相干.\textbf{优}:占用带宽($B=W$)小,能耗低;\textbf{缺}:实现复杂\\
    \textbf{残边带(vestigial-sideband,VSB)调制}:先DSB-SC AM(乘一余弦载波$A_c\cos(2\pi f_ct)$)得$A_cm(t)\cos(2\pi f_ct)$,再带通滤波器(相应函数$H(f)$,使得大部分上/下边带和少部分下/上边带通过),得$x_c(f)=\frac{A_c}{2}[M(f+f_c)+M(f-f_c)]H(f)$.\textbf{检波}:先乘一本地振荡$A_c'\cos(2\pi f_ct)$,得$d(t)=A_c'x(t)\cos(2\pi f_ct)$,$D(f)=\frac{A_c'}{2}[X_c(f+f_c)+X_c(f-f_c)]=\frac{A_cA_c'}{4}\{[H(f-f_c)+H(f+f_c)]+M(f+2f_c)H(f+f_c)+M(f-2f_c)H(f-f_c)\}$,再低通滤波得$Y_D(f)=\frac{A_cA_c'}{4}M(f)[H(f-f_c)+H(f+f_c)]$,\textbf{要求}:$H(f-f_c)+H(f+f_c)=2H(f_c)$,$-W\leq f\leq W$,即$H(f)$在点$(f_c,H(f_c))$周围关于该点对称,$H(f_c+\Delta f)+H(f_c-\Delta f)=2H(f_c)$,$\Delta f\leq W$.所需带宽介于DSB和SSB之间,\textbf{优}:对滤波器要求较SSB放松\\
    \textbf{脉冲幅度调制(pulse amplitude modul,PAM)}:\\%未完
    \textbf{脉冲编码调制(pulse code modul,PCM)}:\\
    \rule{\columnwidth}{.5pt}\\
    \textbf{角度调制}:$x_c(t)=A_c\cos[2\pi f_ct+\phi(t)]$,其中瞬时相位(instantaneous phase)$\theta=2\pi f_ct+\phi(t)$,\textbf{相偏(phase deviation)}$\phi(t)$,\textbf{瞬时频率(instantaneous freq)}:$f(t)=\frac{1}{2\pi}\frac{\mathrm{d}\theta(t)}{\mathrm{d}t}=f_c+\frac{1}{2\pi}\frac{\mathrm{d}\phi(t)}{\mathrm{d}t}$,\textbf{频率偏移(频偏,freq deviation)}$\frac{1}{2\pi}\frac{\mathrm{d}\phi(t)}{\mathrm{d}t}$\\
    \textbf{调相(PM)}:$\phi(t)=k_pm(t)$,$x_c(t)=A_c\cos[2\pi f_ct+k_pm(t)]$,其中$k_p$-\textbf{调相灵敏度(phase deviation const)};\textbf{调频(FM)}:$\frac{\mathrm{d}\phi(t)}{\mathrm{d}t}=k_fm(t)=2\pi f_dm(t)$,$x_c(t)=A_c\cos[2\pi f_ct+2\pi f_d\int_0^tm(\tau)\,\mathrm{d}\tau+\phi_0]$,其中$f_d$-\textbf{调频灵敏度(freq deviation const)};$m(t)$连续,则PM和FM均连续,$m(t)$不连续,则PM不连续,FM依然连续\\
    \textbf{用调相器调频}:先对$m(t)$积分得$\int_0^tm(\tau)\,\mathrm{d}\tau$,再经调相器得$\frac{\mathrm{d}\phi(t)}{\mathrm{d}t}=k_p\frac{\mathrm{d}}{\mathrm{d}t}\int_0^tm(\tau)\,\mathrm{d}\tau=k_pm(t)$.\textbf{用调频器调相}:先对$m(t)$求导得$\frac{\mathrm{d}m(t)}{\mathrm{d}t}$,再经调频器得$\phi(t)=k_f\int_0^t\frac{\mathrm{d}m(\tau)}{\mathrm{d}\tau}\,\mathrm{d}\tau=k_fm(t)$;解调时,先用FM解调器解调得$\frac{\mathrm{d}m(t)}{\mathrm{d}t}$,再积分得$m(t)$,故下仅讨论调频\\
    对AM,$\frac{\mathrm{d}x_c(t)}{\mathrm{d}m(t)}$不依赖$m(t)$,故称\textbf{线性调制(linear modul)},在频域中仅做平移和叠加;对角度调制,$x_c(t)=A_c\cos[2\pi f_ct+\phi(t)]=\re[A_ce^{j2\pi f_ct}e^{j\phi(t)}]=\re\{A_ce^{j2\pi f_ct}[1+j\phi(t)-\frac{1}{2}\phi^2(t)+\cdots]\}$,$\frac{\mathrm{d}x_c(t)}{\mathrm{d}\phi(t)}$依赖$\phi(t)$,故称\textbf{非线性调制(nonlinear modul)},在频域上有非线性变换;当$\phi(t)$很小,高阶相可忽略,从而$\frac{\mathrm{d}x_c(t)}{\mathrm{d}\phi(t)}$不依赖$\phi(t)$,退化为线性调制\\
    \textbf{窄带调频(narrowband FM,NBFM)}:$x_c(t)=A_c\cos[2\pi f_c+k_f\int_0^tm(\tau)\,\mathrm{d}\tau+\phi_0]$,对$m(t)=A_m\cos 2\pi f_mt$,$\phi_0=0$,有$\phi(t)=f_k\int_0^tm(\tau)\,\mathrm{d}\tau=\frac{A_mf_k}{2\pi f_m}\sin 2\pi f_mt=\frac{A_mf_d}{f_m}\sin 2\pi f_mt=\frac{\Delta f}{f_m}\sin 2\pi f_mt=\beta\sin 2\pi f_mt$,其中\textbf{最大频偏(peak freq deviation)}$\Delta f=A_mf_d$,\textbf{调制指数}:$\beta=\text{最大相偏}=\frac{\Delta f}{f_m}=\frac{A_mf_d}{f_m}$;NBFM:$0<\beta\ll 1$,$x_c(t)=A_c\cos[2\pi f_ct+\beta\sin 2\pi f_mt]\approx A_c(\cos 2\pi f_ct-\beta\sin 2\pi f_mt\sin 2\pi f_ct)=A_c\{\cos 2\pi f_ct+\frac{\beta}{2}[\cos 2\pi(f_c+f_m)t-\cos 2\pi(f_c-f_m)t]\}=A_c\re[e^{j\pi f_ct}[1+\frac{\beta}{2}(e^{j2\pi f_mt}-e^{-j\pi f_mt})]]$(作图并对比AM$x_c(t)=A_c\re[e^{j2\pi f_ct[1+\frac{a}{2}(e^{j2\pi f_mt}+e^{-j2\pi f_mt})]}]$即悟角度调制命名原因),包络频谱:$A_c[\frac{1}{2}]$,\textbf{带宽}:$B=2f_m$\\% 包络和相位的频谱图
    \textbf{窄带调频(narrowband PM,NBPM)}:$x_c(t)=A_c\cos[2\pi f_ct+k_pm(t)]$,对$m(t)=A_m\cos 2\pi f_mt$,$\phi_0=0$有,$\phi(t)=k_pA_m\cos 2\pi f_mt=\beta\cos 2\pi f_mt$,其中\textbf{最大频偏}$\Delta f=\max\abs{\frac{1}{2\pi}\frac{\mathrm{d}\phi(t)}{\mathrm{d}t}}=k_pf_mA_m$,\textbf{调制系数(modul index)}:$\beta=\text{最大相偏}=\frac{\Delta f}{f_m}=k_pA_m$,\textbf{带宽}:$B=2W$\\
    \textbf{窄带角度调制}:$x_c(t)=A_c[\cos 2\pi f_ct-\phi(t)\sin(2\pi f_ct)]$,对NBFM,先积分并乘$2\pi f_d$,对NBPM,直接乘$k_p$,得$\phi(t)$,再乘本地振荡$-\sin 2\pi f_ct$,后加上相移$\frac{\pi}{2}$后的本地振荡$\cos 2\pi f_ct$,最后乘$A_c$得$x_c(t)$\\
    \textbf{宽带调频(WBFM)}:$x_c(t)=A_c\cos[2\pi f_ct+\beta\sin 2\pi f_mt]=\re[A_ce^{j2\pi f_ct}e^{j\beta\sin 2\pi f_mt}]$,其中$e^{j\beta\sin 2\pi f_mt}=\sum_{n=-\infty}^{\infty}C_ne^{j2\pi nf_mt}$,$C_m=\frac{1}{T}\int_{-T/2}^{T/2}e^{j\beta\sin 2\pi f_mt}e^{-j2\pi nf_mt}=J_n(\beta)$-一类$n$阶贝塞尔(Bessel)函数(若$n$偶,$\beta_{-n}(\beta)=J_n(\beta)$,若$n$奇,$\beta_{-n}(\beta)=-J_n(\beta)$;$\lim_{\beta\rightarrow 0}J_0(\beta)1$,$\lim_{\beta\rightarrow 0}J_1(\beta)=\frac{\beta}{2}$,$\lim_{\beta\rightarrow 0}J_n(\beta)=0\forall n\geq 2$;$\sum_{n=-\infty}^{\infty}J_n^2(\beta)=1$);\textbf{卡森公式(Carson's rule)}:\textbf{带宽}$B\approx 2(\Delta +f_m)=2(1+\beta)f_m=2(1+\frac{1}{\beta})\Delta f$,对大$\beta$,$B\approx 2nf_m$,对小$\beta$,$B\approx 2f_m$;平均功率:$P=\frac{A_c^2}{2}\sum_{n=-\infty}^{\infty}J_n^2(\beta)=\frac{A_c^2}{2}$,功率不变,仅在频域上根据$\beta$重新分配;对任意$m(t)$,\textbf{偏移系数(deviation ratio)}$D=\frac{f_d\max\abs{m(t)}}{W}$,\textbf{卡森公式}:\textbf{带宽}:$B\approx 2(1+D)W$,其中$W$-$m(t)$的带宽\\
    \textbf{WBFM的产生}:\textbf{方法1:}直接用压控振荡器(voltage-controlled oscillator,VCO)调节载波频率,\textbf{要求}:需要稳频措施,只能产生很小的$\beta$($<0.2$),\textbf{方法2:阿姆斯特朗间接调频发射器(Armstrong indirect FM transmitter)}:先经NBFM系统得一NBFM信号$x(t)=A_c\cos[2\pi f_0t+\phi(t)]$,再经$n$倍倍频器($\times n$ freq multiplier)得$y(t)=A_c\cos[2\pi nf_0t+n\phi(t)]$,此时载波频率,最大频偏和偏移系数$D$均变为原来的$n$倍,再与本地振荡$2\cos 2\pi f_{\text{LO}}$相乘得$e(t)=2A_c\cos[2\pi nf_ct+n\phi(t)]\cos 2\pi f_{\text{LO}}=A_c\{\cos[2\pi(nf_0+f_{\text{LO}})t+n\phi(t)]+\cos[2\pi(nf_0-f_{\text{LP}})t+n\phi(t)]\}$,最后带通滤波(中心频率$f_c=nf_0+f_{\text{LO}}$或$nf_0-f_{\text{LO}}$,带宽由卡森公式得)得$x_c(t)=A_c\cos[2\pi f_ct+n\phi(t)]$\\
    \textbf{WBFM的解调}:\textbf{直接方法}:鉴频器(freq discriminator),输入$x_r(t)=A_c[2\pi f_ct+2\pi f_d\int_0^tm(\tau)\,\mathrm{d}\tau+\phi_0]$,输出$y_D(t)=K_Df_dm(t)$,其中$K_D$-鉴频常数(discriminator const),鉴频器的一种实现方法:先微分得$\frac{\mathrm{d}x_r(t)}{\mathrm{d}t}=-A_c[2\pi f_ct+2\pi f_dm(t)]\sin[2\pi f_ct+2\pi f_d\int_0^tm(\tau)\,\mathrm{d}\tau+\phi_0]$,再包络检波得$A_c[2\pi f_c+2\pi f_dm(t)]$,其中要求$f_c>-f_dm(t)\forall t$,最终输出$y_D(t)=2\pi A_cf_dm(t)$;\textbf{间接方法}:锁相环,$x_r(t)=A_c\cos[2\pi f_ct+\phi(t)]$输入鉴相器(phase detector,输出正比两个输入的相位差),鉴相器的另一个输入为$e_0(t)=A\sin[2\pi f_ct+\theta(t)]$,输出为$e_d(t)\propto\text{Lp}[x_r(t)e_0(t)]\propto=\frac{1}{2}A_cA_vK_d\sin[\phi(t)-\theta(t)]$,当$\phi(t)-\theta(t)$很小,$e_d(t)\propto\frac{1}{2}A_cA_vK_d[\phi(t)-\theta(t)]$,经过环滤波器(loop filter)和环放大器(loop amplifier)(两者总响应函数$F(f)$)得输出$e_v(t)=e_d(t)*f(t)$,输出分出一部分输入VCO得$e_0(t)$作为鉴相器的输入,其频偏$\frac{\mathrm{d}\theta(t)}{\mathrm{d}t}=K_ve_v(t)=\frac{1}{2}A_cA_vK_dK_v[\phi(t)-\theta(t)]*f(t)$,$jf\theta(f)=\mathscr{F}[\frac{\mathrm{d}\theta}{\mathrm{d}t}]=K_t\psi(t)F(f)$,其中$\psi(t)=\phi(t)-\theta(t)$,环路增益$K_t=\frac{1}{2}A_cA_vK_dK_v$,从而$jf[\phi(t)-\psi(t)]=K_t\psi(f)F(f)$,$\psi(f)=\frac{jf\phi(f)}{K_tF(f)+jF}=\frac{1}{1+L(f)}\phi(f)$,当$K_t\rightarrow\infty$,$L(f)=\frac{K_tF(f)}{jf}\rightarrow\infty$,$\psi(f)\rightarrow 0$,$\phi(t)\approx\theta(t)=K_d\int_0^te_v(t)\,\mathrm{d}t$,故$e_v(t)=\frac{1}{K_d}\frac{\mathrm{d}}{\mathrm{d}t}\phi(t)$\\
    \textbf{频分复用}:将不同型号分别调制到不同的频域后再用射频(radio freq,RF)二次调制更高的频域进行传输;\textbf{解调}:先RF解调,再带通滤波得所需频段信号,最后用对应方式二次解调\\% 略\\
    \rule{\columnwidth}{.5pt}
\end{multicols}
\end{document}